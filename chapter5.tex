%----------------------------------------------------------------------------
\chapter{Conclusion and future work}\label{sect:Future}
%----------------------------------------------------------------------------
In my opinion we only started to explore the possibilities of this technique in the realms of natural language processing research and there are still a lot of open questions. The results show that this model trained on the extracted summaries can achieve higher ROUGE score than the benchmark gensim TextRank based algorithm.

My plans for the future include further experimenting with the optimization, modifying the graph attention, and trying out new summary reconstruction methods and new structures.

Modifications of the graph attention layer may include taking into account the edge types in some way because the current one only accounts for the node connectivity. I might reintroduce a modified edge loss for the training process but experiments are needed to determine whether it would be beneficial for the network, since previous iterations suggested otherwise.

In conclusion I think there is room for further development and this research area is highly promising. I would like to delve into graph neural networks even more now that I've come to know them better.